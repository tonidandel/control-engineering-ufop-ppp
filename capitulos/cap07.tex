\chapter{Infraestrutura} 
\label{cap:07} \index{metodologias}
%
\section{Laboratórios}
\begin{itemize}
	\item Laboratório de Mecânica dos Fluidos e Acionamentos Fluido-Mecânicos $80m^{2}$;
	\item Laboratório de Propriedades Mecânicas e Metalografia $63m^{2}$;
	\item Laboratório de Projetos Mecânicos e Usinagem CNC $132m^{2}$;
	\item Laboratório de Tecnologias Industriais $139m^{2}$;
	\item Laboratório de Análise Experimental e Simulação de Sistemas $35m^{2}$;
	\item Laboratório de Eletrônica Analógica e Digital $69m^{2}$;
	\item Laboratório de Protótipos e Desenvolvimento de Novas Tecnologias $56m^{2}$;
	\item Laboratório de Máquinas Elétricas e Interfaceamento $117m^{2}$;
	\item Laboratório de Sistemas Térmicos, Metrologia e Instrumentação $93m^{2}$.
\end{itemize}

\section{Bibliotecas da UFOP}
O discente terá à sua disposição o sistema de bibliotecas da UFOP, o qual é gerenciado pelo sistema SISBIN (Sistema de Informação de Bibliotecas). Poderá também consultar e retirar livros em qualquer biblioteca do sistema.

A principal biblioteca disponível para os discentes é a Biblioteca da Escola de Minas (EM), a qual dispõe de títulos na área básica de Engenharia Mecânica, Programação, Engenharia de Controle e Automação, Processos de Fabricação, Engenharia Metalúrgica, Engenharia de Minas, Engenharia Civil e Engenharia de Produção, contendo acervo atualizado, com mais de $13.518$ títulos e $27.518$ exemplares, e condizente em número e conteúdo com as disciplinas e linhas de pesquisa propostas, e estatística mensal de $3.800$ empréstimos.

O sistema de bibliotecas da UFOP conta ainda com a Biblioteca do Instituto de Ciências Exatas e Biológicas (ICEB), criada em $1982$, ocupando hoje uma área total de $1050 m^{2}$, distribuída em dois andares, com quinze cabines de estudos individuais e salas de estudo em grupo, mais seis computadores destinados aos usuários e doze computadores no total, com acervo de aproximadamente $8.502$ títulos e $22.433$ exemplares, e estatística mensal de $15.000$ empréstimos.

Ambas as bibliotecas dispõem de um espaço amplo e bem ventilado para atender aos alunos, com salas de estudo individuais e em grupo. As bibliotecas em conjunto dispõem de mais de 1.200 títulos nas áreas acima citadas.

O sistema também conta com a Biblioteca Prof. Luciano Jacques de Moraes (DEGEO/DEMIM) e a Biblioteca de Obras Raras da Escola de Minas – BIBORAR. Aquela possui acervo de $12.000$ livros e $90$ títulos de periódicos nacionais e internacionais, e conta ainda com uma mapoteca que disponibiliza cerca de $2.600$ mapas topográficos e geológicos. Esta reúne cerca de $22000$ volumes de publicações técnico-científicas nas áreas de ciências naturais, puras e aplicadas, que incluem livros e periódicos raros, enciclopédias, guias, manuais e legislação, editados entre os séculos XVII ao XX, no Brasil e no exterior. A Biblioteca guarda ainda a Coleção Carlos Walter e a Coleção Ex-alunos e Ex-professores da Escola de Minas, acervos bibliográficos de renomados profissionais que passaram pela instituição.

A Universidade Federal de Ouro Preto faz parte da rede do Portal de Periódicos CAPES. Desta forma, os estudantes terão acesso a textos completos, incluindo os periódicos e anais de congressos da ACM (Association for Computing Machinery) e os periódicos e anais de congressos do IEEE (Institute of Electrical and Electronics Engineers).

Os estudantes poderão utilizar também os laboratórios de informática da Escola de Minas e do Departamento de Ciência da Computação (DECOM). O primeiro laboratório, compartilhado com os cursos instalados na Escola de Minas, possui área física aproximada de $60m^{2}$, e conta com $30$ computadores. Neste laboratório, monitores atendem os alunos fora do horário de aulas. Já o laboratório do DECOM é de uso comum para os cursos de Engenharia, e possui capacidade para atendimento de $60$ alunos. O laboratório também é equipado com datashow, e disponibiliza monitores para atendimento aos alunos fora do horário de aulas.

\section{NITE}
Em 2001 a UFOP criou o Núcleo de Inovação Tecnológica e Empreendedorismo (NITE), com o intuito de promover a formação de um ambiente cooperativo que conjugue interesses da UFOP, empresas e órgãos para promoção de atividades inovadoras e de transferência de tecnologia, com vistas a contribuir para o desenvolvimento social e econômico da região de influência e da Instituição.

Os principais objetivos do NITE são captar e proteger os ativos de propriedade intelectual gerados na UFOP, formar parcerias com empresas e com organizações, a fim de transferir esses ativos ao mercado para o uso público e para o desenvolvimento econômico e implementar a cultura empreendedora no ambiente acadêmico como um todo.

Atualmente, o NITE está dividido em seis setores distintos, pois visa facilitar a implementação das ações propostas, direcionando os planos e ações para cada setor específico.