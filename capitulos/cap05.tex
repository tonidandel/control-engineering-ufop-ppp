\chapter{Metodologias de ensino-aprendizagem} 
\label{cap:05} \index{metodologias}
Os professores do Curso de Engenharia de Controle e Automação da UFOP, exploram os recursos didáticos disponibilizados pela instituição buscando sempre a melhor forma de passar o conteúdo de suas disciplinas e adaptando, sempre que possível, a metodologia de ensino às novas tendências didáticas, comprovadamente eficazes e, muitas vezes, somente aplicáveis a disciplinas de cursos  altamente tecnológicos como é a Engenharia de Controle e Automação. Dentre as metodologias utilizadas destacam-se a Aprendizagem Ativa, o Laboratório Remoto, Aulas Práticas e Aulas Expositivas em que recursos didáticos tais como o projetor, simuladores em computador e o quadro branco são utilizados.   

A  Aprendizagem Ativa...

No uso dos laboratórios em aulas práticas e projetos de pesquisa é comum a falta de equipamentos para todos os alunos conseguirem utilizar ao mesmo tempo e também a falta de tempo para se fazer os experimentos necessários no tempo de duração da aula prática. Para seguir completamente o programa da disciplina, muitas vezes o (a) professor (a) não pode despender de uma segunda ou terceira aula prática para dar continuidade a um experimento e este, muitas vezes, acaba ficando pela metade. O Laboratório Remoto traz uma solução para esses problemas ao possibilitar que o aluno realize a prática a partir de qualquer lugar desde que possua um computador e uma conexão de internet. A prática é fisicamente realizada no Laboratório Remoto, controlado pelas ações do aluno através do sistema, e os resultados são informados ao aluno em tempo real. Dessa forma, cada aluno pode realizar a aula prática, individualmente, no horário que melhor lhe convir. Essa é uma metodologia de ensino do curso que complementa o conteúdo ensinado nas aulas teóricas e que também pode ser utilizada como uma forma de avaliação.
