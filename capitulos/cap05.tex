\chapter{Metodologias de ensino-aprendizagem} 
\label{cap:05} \index{metodologias}
\section{Estratégias de ensino}
Os professores do Curso de Engenharia de Controle e Automação da UFOP, exploram os recursos didáticos disponibilizados pela instituição buscando sempre a melhor forma de passar o conteúdo de suas disciplinas e adaptando, sempre que possível, a metodologia de ensino às novas tendências didáticas, comprovadamente eficazes e, muitas vezes, somente aplicáveis a disciplinas de cursos  altamente tecnológicos como é a Engenharia de Controle e Automação. Dentre as metodologias utilizadas destacam-se a Aprendizagem Ativa, o Laboratório Remoto, Sala de Aula Invertida, Aulas Práticas e Aulas Expositivas em que recursos didáticos tais como o projetor, simuladores em computador e o próprio quadro branco são utilizados.

\subsection*{A aprendizagem ativa}   

A aprendizagem ativa segue a metodologia da Aprendizagem Baseada em Projetos (ABP) na qual o conhecimento vai sendo construído ao poucos, pelos alunos, ao se trabalhar na solução de um problema complexo ou um desafio. Os alunos se envolvem na pesquisa buscando recursos e aplicando seu conhecimento na prática até que se alcance a solução desejada. Tal abordagem incentiva o trabalho em grupo, a criatividade e o interesse pela disciplina, aprimorando o conhecimento tecnológico dos estudantes para projetar, simular e implementar sistemas de controle e automação de processos reais. O (A) docente orienta os trabalhos atuando como um (a) catalisador (a) e não mais como um (a) expositor (a) de conhecimento. 

\subsection*{Sala de aula invertida}
%
Na sala de aula invertida, nome derivado do termo em inglês \textit{``flipped classroom''}, o professor passa a ser um guia que ajuda o aluno a aprender. Ele não mais dita o quê, quando e onde o aluno deve aprender. Nessa abordagem não existe mais o formato padrão do professor à frente da sala expondo o conhecimento aos alunos, mas sim, um professor que circula pela sala retirando dúvidas dos alunos que trabalham na solução de problemas já aplicando o conhecimento que adquiriram fora da sala de aula. Na aquisição desse conhecimento os alunos podem utilizar de conteúdo audiovisual e leituras recomendadas/produzidas pelo professor, as quais podem ser complementadas pelo conteúdo de fontes confiáveis buscadas pelos próprios alunos.

Na sala de aula invertida o processo de aprendizagem se torna mais humanizado, com o(a) professor(a) mais próximo (a) aos alunos e com esses tendo a oportunidade de escolher como e onde eles aprenderão um determinado assunto. A avaliação é feita de forma contínua pelo desempenho dos alunos na resolução das atividades individuais ou em grupo, realizadas em sala ou laboratório.  Além disso, tal abordagem explora as diversas fontes de conhecimento e recursos tecnológicos disponíveis atualmente, constituindo uma metodologia de ensino moderna e atual.

\subsection*{Aulas práticas}
%
As aulas práticas ministradas nos diversos laboratórios da UFOP são parte de muitas disciplinas obrigatórias e eletivas do curso de Engenharia de Controle e Automação e constituem uma forma de complementar e fixar o conteúdo teórico visto em sala de aula. Os laboratórios também desempenham papel fundamental ao possibilitar o desenvolvimento de muitos projetos de pesquisa e monografias. Em alguns laboratórios, o(a) docente e o(a) discente contam com o auxílio de um(a) técnico(a) administrativo(a) que ajuda na organização do laboratório, da aula prática e pode também auxiliar na execução de algumas tarefas no decorrer da prática ou do projeto de pesquisa/monografia. Nas aulas práticas, o(a) estudante tem a oportunidade de ter uma experiência similar àquela que terá no mercado de trabalho bem como desenvolver habilidades impossíveis de ser ensinadas apenas em teoria.

No uso dos laboratórios em aulas práticas e projetos de pesquisa é comum, por uma diversa gama de fatores, a falta de equipamentos para que todos os alunos possam usufruir os recursos disponíveis simultaneamente, além da falta de tempo para se realizar os experimentos necessários no tempo de duração da aula prática. Para seguir completamente o programa da disciplina, muitas vezes o(a) docente não pode prescindir de uma segunda ou terceira aula prática para dar continuidade a um experimento e este, muitas vezes, acaba ficando pela metade. O Laboratório Remoto traz uma solução para esses problemas ao possibilitar que o aluno realize a aula prática em qualquer lugar, desde que possua um computador e uma conexão com a internet. 

A aula prática é fisicamente realizada no Laboratório Remoto, controlado pelas ações do aluno por meio do sistema e os resultados são informados ao aluno em tempo real. Dessa forma, cada estudante pode realizar a aula prática individualmente, no horário que melhor lhe convir. Essa é uma metodologia de ensino do curso que complementa o conteúdo ensinado nas aulas teóricas e que também pode ser utilizada como uma forma de avaliação. O curso de Engenharia de Controle e Automação da UFOP possui um laboratório remoto próprio e mais outros estão em desenvolvimento. Ademais, alguns laboratórios remotos presentes em outras universidades pelo Mundo podem ser acessados gratuitamente.   

Nas aulas expositivas o(a) docente conta com recursos tais como o projetor e softwares de simulação que possibilitam enriquecer as aulas teóricas e práticas com vídeos, figuras elaboradas, animações e demonstrações de como utilizar softwares (de programação e simulação, por exemplo) adequadamente, no decorrer das disciplinas do curso. 

Com os softwares de simulação o(a) docente pode ainda exemplificar conteúdos que serão tratados nas aulas práticas ou realizar testes que não são possíveis mesmo nas aulas práticas pela falta de algum componente ou equipamento, bem como complementar o conteúdo da teoria.

Aulas expositivas também são ministradas da forma tradicional com a utilização do quadro branco que é prática comum em muitas disciplinas ditas ``teóricas''.

Por fim, atenta-se para a revolução tecnológica a qual o Mundo vem passando nos últimos anos, com novas tecnologias sendo criadas em curtos intervalos de tempo e grande acesso à informação, o que gera a necessidade de adaptação contínua por parte dos profissionais da educação, que trabalham diretamente com tais tecnologias. Dessa forma, os métodos tradicionais de ensino também precisam evoluir para formar esse novo profissional, pronto para enfrentar os desafios que essa nova era traz. Cientes dessa necessidade, os professores e técnicos do curso de Engenharia de Controle e Automação da UFOP trabalham continuamente em adaptar ou desenvolver as mais modernas metodologias de ensino. 

\section{Projetos Futuros}
Apresenta-se também algumas ideias e projetos que cogita-se desenvolver a médio e longo prazos, no âmbito da atuação dos docentes do curso de Engenharia de Controle e Automação da UFOP.

\subsection*{OpenLab - O laboratório de Engenharia ``Opensource''}

O \textit{OpenLab} consistirá em um laboratório de criatividade, aprendizado e inovação acessível a todos interessados em criar, desenvolver e construir projetos. Através de processos colaborativos de criação, compartilhamento do conhecimento e do uso de ferramentas de fabricação digital, o \textit{OpenLab}, inspirado em iniciativas já existentes como os \textit{FabLabs} do estado de São Paulo, visará trazer à população a possibilidade de aprender, projetar e  produzir diversos tipos de objetos, e em diferentes escalas.

Os laboratórios deverão ser equipados com impressoras 3D, cortadoras a laser, plotter de recorte, fresadoras CNC, computadores com software de desenho digital CAD, todos com a filosofia \textit{Open Source}, equipamentos de eletrônica, robótica e ferramentas de marcenaria e mecânica. A ideia é estruturar o \textit{OpenLab} com uma equipe dinâmica que incentiva o aprendizado compartilhado e a criatividade através do fazer, realizando cursos e orientando o desenvolvimento de projetos.

Pretende-se oferecer oficinas, cursos e palestras, disseminando a produção do conhecimento em tecnologia, ciência, arte e inovação. Através de um processo humanizado, as atividades de ensino visarão estimular o compartilhamento da informação e construção coletiva de ideias. A ideia do \textit{OpenLab} é democratizar o acesso às tecnologias de fabricação digital, disponibilizando ferramentas tecnológicas de última geração e vivência em grupo em um ambiente colaborativo e inovador.
