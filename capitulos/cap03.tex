\chapter{Administra{\c c}{\~a}o Acad{{\^e}mica}}
\label{cap:03} \index{administracao}

\section{Coordenação Acadêmica}
Segundo o Estatuto e Regimento da Universidade Federal de Ouro Preto-UFOP, de $1999$ (Artigo $23$, $ 1^{\circ}$  parágrafo), ``os Colegiados de Cursos são as instâncias universitárias responsáveis pela coordenação didática das disciplinas constituintes do projeto pedagógico de cada curso''.

A coordenação acadêmica do curso de Engenharia de Controle e Automação é exercida pelo Colegiado de Curso de Engenharia de Controle e Automação – CECAU, que é presidido pelo coordenador do curso, cuja composição e atribuições cumprem o que determina o Estatuto (Resolução CUNI $n. 414$ $-$ atualizada em $28$ de agosto de $2014$)(anexo \ref{ape:02}) e o Regimento Geral (Resolução CUNI nº 435 – atualizada em 22 de dezembro de 2014) da Universidade Federal de Ouro Preto, conforme a Portaria nº 1.486, de 29 de dezembro de 1998, publicado, no Diário Oficial da União de 30 de dezembro de 1998, seção I, página 12. Como órgão deliberativo, cabe ao colegiado:

\begin{itemize}
	\item avaliar o Projeto Político e Pedagógico, que é elaborado pelo Núcleo Docente Estruturante (NDE);
	\item compatibilizar as diretrizes gerais dos programas das disciplinas do respectivo curso e determinar aos departamentos as modificações necessárias;
	\item estabelecer as normas para a orientação acadêmica; 
	\item indicar para a Pró-Reitoria de Graduação os candidatos à colação de graus. 
\end{itemize}

Os Colegiados de Curso de Graduação são constituídos por representantes dos Departamentos que oferecem as disciplinas do Curso, eleitos pelas respectivas Assembleias de Departamento, em proporção ao número de créditos das disciplinas ministradas, na forma do Regimento Geral da UFOP, com mandato de dois anos, permitida uma recondução. O limite máximo de representante por departamento é de quatro membros. A representação estudantil obedece o Regimento Geral da UFOP e tem assento assegurado no colegiado.

Visando ao acompanhamento permanente das atividades inerentes ao curso, o Colegiado do curso lança mão de diversas ações, dentre as quais os dados levantados pela Pró-Reitoria de Graduação (PROGRAD) e outros órgãos da UFOP. A PROGRAD dispõe de um Programa de Avaliação das Disciplinas e do Trabalho dos Docentes que realiza periodicamente levantamentos acerca do desenvolvimento dos cursos de graduação, analisando: o fluxo de alunos, as taxas de aprovação/reprovação nas disciplinas, a taxa de evasão, entre outros. Com relação ao desenvolvimento das disciplinas, é aplicado semestralmente um conjunto de questionários a docentes e discentes, cujos resultados são organizados e disponibilizados para a comunidade universitária.  

\subsection*{Objetivos da avaliação das disciplinas e do trabalho docente}

\textbf{Geral}

\begin{itemize}
\item Buscar a permanente melhoria da qualidade dos cursos de graduação.
\end{itemize}

\textbf{Específicos}

\begin{itemize}
\item orientar o professor responsável pela disciplina quanto ao desempenho de suas atividades de ensino;
\item fornecer elementos aos colegiados de cursos, departamentos, etc., no que se refere à permanente avaliação e planejamento de seu Projeto Político Pedagógico;
\item criar um canal institucional de participação dos estudantes e professores na melhoria da qualidade do ensino e projetos político pedagógicos de seus cursos;
\item levantar elementos que permitam uma comparação saudável entre o conjunto das disciplinas avaliadas, dos departamentos e dos cursos da Instituição.
\end{itemize}

\subsection*{Atribuições do Coordenador $-$ (Presidente do Colegiado)}

Além das atividades inerentes à coordenação acadêmica e didática das disciplinas constituintes do projeto pedagógico do curso, o Presidente do Colegiado do Curso de Engenharia de Controle e Automação, além de outros setores da Universidade, deverá manter contatos com empresas, institutos de pesquisa, repartições públicas, etc. com o objetivo de estabelecer os convênios para realização de estágios, pesquisa, monitoria e extensão a serem realizados pelos alunos do curso. Em seguida, explicita-se o nome do presidente e da secretária do colegiado de curso.

\textbf{CECAU - Colegiado do Curso de Engenharia de Controle e Automação}

Presidente: Prof. Agnaldo José da Rocha Reis

E-mail: agnreis@gmail.com

\textbf{Secretária: Rosilene Pedrosa Gomes}

Telefone: 3559-1542

E-mail: col.ufop@gmail.com

Além das atividades inerentes à coordenação acadêmica e didática das disciplinas constituintes do projeto pedagógico do curso, o Presidente do Colegiado do Curso de Engenharia de Controle \& Automação, além de outros setores da Universidade, deverá manter contatos com empresas, institutos de pesquisa, repartições públicas, etc. com o objetivo de estabelecer os convênios para realização de estágios, pesquisa, monitoria e extensão a serem realizados pelos alunos do curso.

\section{Atribuições do colegiado de curso}
O Artigo $25$ do atual Estatuto e Regimento vigente estabelece que os colegiados de curso da UFOP têm as seguintes atribuições:
\begin{enumerate}
	\item Compatibilizar as diretrizes gerais dos programas das disciplinas do respectivo curso e determinar aos departamentos as modificações necessárias;
	
	\item Integrar os planos elaborados pelos departamentos, relativos ao ensino das várias disciplinas, para fim de organização do programa didático do curso;
	
	\item Recomendar ao departamento, a que esteja vinculada a disciplina, as providências adequadas à melhor utilização das instalações, do material e do aproveitamento do pessoal;
	
	\item Propor a aprovação do Conselho de Ensino Pesquisa e Extensão o currículo pleno do curso e suas alterações, com indicação dos pré-requisitos, da carga horária, das ementas, dos programas e dos créditos das disciplinas que o compõem;
	
	\item Decidir sobre questões relativas a reopção de cursos, equivalência de disciplinas, jubilamento, matrícula em disciplinas isoladas e transferência;
	
	\item Apreciar as recomendações dos departamentos e requirimentos dos doscentes sobre assunto de interesse do curso;
	
	\item Exercer atividades de orientação acadêmica dos estudantes do curso, com vistas ao cumprimento dos créditos necessários à candidatura à colação de grau;
	
	\item Indicar para a Pró-Reitoria de Graduação os candidatos à colação de grau.
\end{enumerate}

\subsection*{Outros colegiados}

\textbf{COARQ - Colegiado do Curso de Arquitetura e Urbanismo}

Presidente: Profa. Sandra Maria Antunes Nogueira

E-mail: sandramnog@gmail.com

Secretária: Naiara Pinheiro de Castilho

Telefone: 3559-1542

E-mail: colegiados.em@gmail.com


\textbf{CEAMB - Colegiado do Curso de Engenharia Ambiental}

Presidente: Prof. Gilberto Queiroz da Silva

E-mail: gqueiroz@em.ufop.br

Secretária: Marilene Guimarães Bretas

Telefone: 3559-1542

E-mail: ceamb@em.ufop.br

\textbf{CECIV - Colegiado do Curso de Engenharia de Civil}

Presidente: Prof. Geraldo Donizetti de Paula

E-mail: gdepaula9@gmail.com

Secretária: Marilene Guimarães Bretas

Telefone: 3559-1542

E-mail: ceciv@em.ufop.br

\textbf{CECAU - Colegiado do Curso de Engenharia de Controle e Automação}

Presidente: Prof. Agnaldo José da Rocha Reis

E-mail: agnreis@gmail.com

Secretária: Rosilene Pedrosa Gomes

Telefone: 3559-1542

E-mail: col.ufop@gmail.com

\textbf{CEGEO - Colegiado do Curso de Engenharia Geológica}

Presidente: Prof. Marco Antônio Fonseca

E-mail: marco@degeo.ufop.br

Secretária: Rosilene Pedrosa Gomes

Telefone: 3559-1542

E-mail: col.ufop@gmail.com

\textbf{CEMEC - Colegiado do Curso de Engenharia Mecânica}

Presidente: Prof. Luis Antônio Bortolaia

E-mail: luis.bortolaia@em.ufop.br

Secretária: Naiara Pinheiro de Castilho

Telefone: 3559-1542

E-mail: colegiados.em@gmail.com

\textbf{CEMET - Colegiado do Curso de Engenharia Metalúrgica}

Presidente: Prof. Luiz Cláudio Cândido

E-mail: candido@em.ufop.br

Secretária: Naiara Pinheiro de Castilho

Telefone: 3559-1542

E-mail: colegiados.em@gmail.com

\textbf{CEMIN - Colegiado do Curso de Engenharia de Minas}

Presidente: Prof. Carlos Alberto Pereira

E-mail: pereiraufop@gmail.com

Secretária: Marilene Guimarães Bretas

Telefone: 3559-1542

E-mail: cemin@em.ufop.br

\textbf{CEPRO - Colegiado do Curso de Engenharia de Produção}

Presidente: Profª. Francisca Diana Ferreira

E-mail: dianaufu@yahoo.com.br

Secretária: Rosilene Pedrosa Gomes

Telefone: 3559-1542

E-mail: col.ufop@gmail.com

\section{Núcleo Docente Estruturante (NDE)}

O Núcleo Docente Estruturante foi um conceito criado pela portaria $n^{\circ} 147$ de $2$ de fevereiro de $2007$, do ministério da educação, com o intuito de qualificar o envolvimento docente no processo de concepção e consolidação de um curso de graduação:
\begin{citacao}
	``Entende-se, então, que todo curso que tem qualidade possui (ainda que informalmente) um grupo de professores que, poder-se-ia  dizer,  é a alma do curso. Em  outras palavras, trata-se de um núcleo docente estruturante.	É importante ainda observar que, dentro da tradição bastante  burocratizante  das instituições de ensino no Brasil, recomendar-se ou, mais ainda, exigir-se a existência de um NDE, tenderia a induzir a definição deste como um 
	órgão deliberativo, o que pode significar a perda da eficácia de suas funções. O NDE  deve ser considerado não como  exigência ou requisito legal, mas como elemento diferenciador da qualidade do curso, no que diz respeito à interseção entre as dimensões do corpo docente e Projeto Pedagógico do Curso.''
\end{citacao}

Por meio da resolução CEPE $n. 4450$, de $29$ de abril de $2011$ (anexo \ref{ape:02}, página \pageref{cepe4450}), página), o Conselho de Ensino, Pesquisa e Extensão da Universidade Federal de Ouro Preto instituiu o Núcleo Docente Estruturante (NDE), nos termos da Resolução CONAES $n. 01/2010$, de 17 de junho de 2010, com o intuito de qualificar o envolvimento docente no processo de concepção e consolidação de um curso de graduação. O NDE terá competência acadêmica de acompanhamento e de atuação nos processos de concepção, consolidação e contínua atualização do projeto pedagógico do curso. 

As ações e deliberações do NDE, que devem ser referendadas pelo colegiado, englobam:
\begin{itemize}
	\item Contribuir na consolidação do perfil profissional do egresso do curso; 
	\item Zelar pela integração curricular interdisciplinar entre as diferentes atividades de ensino constantes no currículo; 
	\item Indicar formas de incentivo ao desenvolvimento de linhas de pesquisa e extensão, oriundas de necessidades da graduação, de exigências do mercado de trabalho e afinadas com as políticas públicas relativas à área de conhecimento do curso; e
	\item Zelar pelo cumprimento das Diretrizes Curriculares Nacionais para os Cursos de Graduação.
\end{itemize}

Os integrantes do NDE são designados por Portaria do Diretor da Unidade Acadêmica responsável pela oferta do curso de graduação, a partir de uma lista de professores indicados pelo Colegiado de Curso, para um mandado de três anos, permitindo-se reconduções sucessivas, caso seja, compreendidas com fator positivo para o curso. É recomendada a manutenção de pelo menos 1/3 dos membros atuais na renovação da composição. Pelo menos $60\%$ dos membros deve ter titulação acadêmica stricto sensu e $20 \%$ dos membros com regime de trabalho em tempo integral e a presidência será exercida por um de seus membros eleito pelos seus pares.

Ao NDE cabe a manutenção do presente Projeto Pedagógico de Curso (PPC) e a correspondente implementação. O NDE é um órgão consultivo, cujas sugestões e decorrentes ações devem ser avaliadas e aprovadas pelo Colegiado de Curso de Engenharia de Controle e Automação, que é o órgão deliberativo. 

Este grupo deve avaliar constantemente o andamento do Curso, especialmente nos primeiros anos, propondo melhorias e ajustes no PPC que impactem no bom funcionamento do Curso, de forma a possibilitar a realização dos objetivos propostos.

\section{Corpo Docente}
O curso de Engenharia de Controle e Automação está instalado na Escola de Minas – EM, sendo vinculado, nas áreas de concentração/ênfase, aos departamentos de:
\begin{itemize}
	\item Engenharia de Controle e Automação – responsável pelas áreas de concentração em Controle de Processos Industriais e em Automação de Processos;     
	\item Computação (DECOM) – participação, com oferecimento de disciplinas específicas, nas áreas de concentração citadas acima;
	\item Engenharia de Minas (DEMIN) – participação, com oferecimento de disciplinas específicas, na área de concentração em Controle de Processos Industriais: Mineração e Metalurgia;
	\item Engenharia Metalúrgica e de Materiais (DEMET) - participação, com oferecimento de disciplinas específicas, na área de concentração em Controle de Processos Industriais: Mineração e Metalurgia.
	\item Engenharia Mecânica (DEMET) – participação, com oferecimento de disciplinas específicas, nas áreas de concentração de engenharia mecânica; 
\end{itemize} 

Além destes departamentos há a participação de outros Departamentos da UFOP, como o Departamento de Matemática (DEMAT), o Departamento de Física (DEFIS), o Departamento de Química (DEQUI), pertencentes ao Instituto de Ciências Exatas e Biológicas, o Departamento de Educação (DEEDU), pertencente ao Instituto de Ciências Humanas e Sociais, o Departamento de Filosofia (DEFIL), pertencente ao Instituto de Filosofia, Arte e Cultura, o Departamento de Direito (DEDIR), e também dos Departamentos de Engenharia de Produção, Administração e Economia (DEPRO), Arquitetura e Urbanismo (DEARQ) e Engenharia Civil (DECIV) da Escola de Minas, que oferecem disciplinas do Curso de Engenharia de Controle e Automação.

Nas tabelas 2 e 3 destaca-se a relação de professores efetivos e de técnicos administrativos de Departamento de Engenharia de Controle e Automação (DECAT).  

O corpo docente pode ser consultado a partir da tabela \ref{tab:0301} e o corpo de técnicos administrativos pode ser consultado na tabela \ref{tab:0302} .

% Please add the following required packages to your document preamble:
% \usepackage{graphicx}
% \usepackage[table,xcdraw]{xcolor}
% If you use beamer only pass "xcolor=table" option, i.e. \documentclass[xcolor=table]{beamer}
\begin{table}[p]
		\caption{Corpo Docente - Engenharia de Controle \& Automação e Engenharia Mecânica.}
		\label{tab:0301}
	\resizebox{\textwidth}{!}{%
		\begin{tabular}{|l|l|l|}
			\hline
			\multicolumn{1}{|c|}{{\color[HTML]{3531FF} \textbf{Nome}}} & \multicolumn{1}{c|}{{\color[HTML]{3531FF} \textbf{Área}}} & \multicolumn{1}{c|}{{\color[HTML]{3531FF} \textbf{Sub-área}}} \\ \hline
			{\color[HTML]{3531FF} \textbf{Adrielle de Carvalho Santana}} & Controle e Automação & \begin{tabular}[c]{@{}l@{}}Sistemas eletrônicos de Medida e Controle\\ Medição, Controle, Correção e Proteção \\ de Sistemas Elétricos de Potência\\ Eletrônica Industrial, Sistemas e Controles Eletrônicos\\ Controle de Processos Eletrônicos, Retroalimentação.\end{tabular} \\ \hline
			{\color[HTML]{3531FF} \textbf{Agnaldo Jose da Rocha Reis}} & Controle e Automação & \begin{tabular}[c]{@{}l@{}}Instrumentação,\\ Interfaceamento de sistemas\\ Laboratórios de Controle e Automação\\ Identificação de Sistemas.\end{tabular} \\ \hline
			{\color[HTML]{3531FF} \textbf{Alan Kardek Rego Segundo}} & Controle e Automação & \begin{tabular}[c]{@{}l@{}}Sistemas eletrônicos de Medida e Controle \\ Controle de Processos Eletrônicos,\\ Retroalimentação\\ Instrumentação eletrônica.\end{tabular} \\ \hline
			{\color[HTML]{3531FF} \textbf{Danny Augusto Vieira Tonidandel}} & Engenharia Elétrica & \begin{tabular}[c]{@{}l@{}}História da Engenharia Elétrica\\ Matemática Aplicada à Engenharia \\ Ensino de Engenharia Elétrica \end{tabular} \\ \hline
			{\color[HTML]{3531FF} \textbf{João Carlos Vilela de Castro}} & Controle e Automação & Robótica \\ \hline
			{\color[HTML]{3531FF} \textbf{José Alberto Naves Cocota Junior}} & Controle e Automação & \begin{tabular}[c]{@{}l@{}}Sistemas eletrônicos de medida e controle\\ Controle de Processos Eletrônicos, Retroalimentação\\ Instrumentação eletrônica.
			\end{tabular} \\ \hline
			{\color[HTML]{3531FF} \textbf{Karla Boaventura P. Palmieri}} & Controle e Automação & \begin{tabular}[c]{@{}l@{}}Informática Industrial\\ Sistemas Integrados de Manufatura\end{tabular} \\ \hline
			{\color[HTML]{3531FF} \textbf{Luciana Gomes Castanheira}} & Controle e Automação & Robótica. \\ \hline
			{\color[HTML]{3531FF} \textbf{Luiz Fernando Ríspoli Alves}} & Controle e Automação & Automação Residencial e Predial \\ \hline
			{\color[HTML]{3531FF} \textbf{Paulo Marcos de Barros Monteiro}} & Controle e Automação & Sistemas de Controle. \\ \hline
			{\color[HTML]{3531FF} \textbf{Regiane de Sousa Silva Ramalho}} & Controle e Automação & \begin{tabular}[c]{@{}l@{}}Automação eletrônica de processos\\ elétricos e industriais.\end{tabular} \\ \hline
		\end{tabular}%
	}
\end{table}




% Please add the following required packages to your document preamble:
% \usepackage{graphicx}
% \usepackage[table,xcdraw]{xcolor}
% If you use beamer only pass "xcolor=table" option, i.e. \documentclass[xcolor=table]{beamer}
\begin{table}[]
	\centering
	\caption{Corpo de técnicos administrativos em educação (TAE).}
	\label{tab:0302}
	%\resizebox{\textwidth}{!}{%
		\begin{tabular}{lll}
			\hline
			\multicolumn{1}{c}{{\color[HTML]{3531FF} \textbf{Nome}}}                                                                           & \multicolumn{1}{c}{{\color[HTML]{3531FF} \textbf{Área}}} & \multicolumn{1}{c}{{\color[HTML]{3531FF} \textbf{Cargo}}} \\ \hline
			\multicolumn{1}{|l|}{{\color[HTML]{3531FF} \textbf{\begin{tabular}[c]{@{}l@{}}Diógenes\\ 			Viegas Mendes Ferreira\end{tabular}}}} & \multicolumn{1}{l|}{Controle e Automação}                & \multicolumn{1}{l|}{Técnico efetivo}                      \\ \hline
			\multicolumn{1}{|l|}{{\color[HTML]{3531FF} \textbf{\begin{tabular}[c]{@{}l@{}}Fernando\\ 			dos Santos Alves\end{tabular}}}}       & \multicolumn{1}{l|}{Controle e Automação}                & \multicolumn{1}{l|}{Técnico efetivo}                      \\ \hline
			\multicolumn{1}{|l|}{{\color[HTML]{3531FF} \textbf{\begin{tabular}[c]{@{}l@{}}Francisco\\ 			de Paula Coelho\end{tabular}}}}       & \multicolumn{1}{l|}{Controle e Automação}                & \multicolumn{1}{l|}{Técnico efetivo}                      \\ \hline
			\multicolumn{1}{|l|}{{\color[HTML]{3531FF} \textbf{\begin{tabular}[c]{@{}l@{}}Robson\\ 			Nunes Dal Col\end{tabular}}}}            & \multicolumn{1}{l|}{Controle e Automação}                & \multicolumn{1}{l|}{Técnico efetivo}                      \\ \hline
			\multicolumn{1}{|l|}{{\color[HTML]{3531FF} \textbf{\begin{tabular}[c]{@{}l@{}}Roberta\\ 			Kelly Barbosa\end{tabular}}}}           & \multicolumn{1}{l|}{Administrativa}                      & \multicolumn{1}{l|}{Técnica efetiva}                      \\ \hline
		\end{tabular}%
	%}
\end{table}

\section{Atividades Complementares, Estágio e Trabalho de Conclusão de Curso}
As atividades complementares englobam as atividades acadêmicas desenvolvidas pelos alunos ligadas a programas de pesquisa, monitoria e extensão da UFOP. Essas atividades complementares, para o estudante de Engenharia de Controle e Automação da UFOP, serão regidas por normas específicas da UFOP, recebendo a concessão de créditos conforme Resolução CEPE $n^{\circ} 1.987$ e obedecendo critérios estabelecidos pelo Colegiado do Curso (consultar anexo \ref{ape:02}). Essas atividades devem ser desenvolvidas, preferencialmente, em uma área da Engenharia de Controle e Automação e serão incentivadas pelo Professor Orientador Acadêmico. 

No último ano o aluno de Engenharia de Controle e Automação da UFOP deverá matricular-se nas disciplinas Trabalho Final de Curso I e II e, sob a orientação de um professor, desenvolverá um trabalho, na área para a qual fez opção para aprofundar seus estudos, o que dará origem à monografia que será defendida perante comissão examinadora, no final do décimo período, obtendo o aluno 30 (trinta) créditos.