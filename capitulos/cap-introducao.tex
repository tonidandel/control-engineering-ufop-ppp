\chapter{Apresenta{\c c}{\~a}o} 
\label{cap:introducao} \index{Introdu{\c c}{\~a}o}
%\addcontentsline{toc}{chapter}{Introdu{\c c}{\~a}o}%%
%% ------------------------------------------------------------------------- %%
%\label{sec:consideracoes_preliminares}
Apresenta-se uma descri{\c c}{\~a}o do processo de constru{\c c}{\~a}o do projeto pedag{\'o}gico para o curso de gradua{\c c}{\~a}o em engenharia de controle e automa{\c c}{\~a}o da UFOP, de acordo com as orienta{\c c}{\~o}es do plano de desenvolvimento institucional (PDI) e do projeto pedagógico institucional (PPI) da UFOP \cite{manual-ppc-ufop}. 

Com o objetivo de promovermos ampla divulgação e possibilitar a participação ativa (de professores, estudantes, egressos, profissionais e da comunidade em geral) nos princípios norteadores do curso, disponibilizamos este Projeto Pedagógico de Curso em um repositório do GitHub: <\url{https://github.com/tonidandel/ppc-eca-ufop}>.
\index{objetivos!objetivo} % index permite acrescentar um item no indice remissivo
%% 
\section{A Universidade Federal de Ouro Preto}\label{sec:contextualizacao}
A Universidade Federal de Ouro Preto foi instituída como Fundação de Direito Público em 21 de agosto de 1969, incorporando as escolas de Farmácia e de Minas, instituições de ensino superior criadas no século XIX. Com a incorporação das duas tradicionais escolas, as raízes da UFOP remontam a 04 de abril de 1839, a partir da aprovação, pela Assembléia Legislativa de Minas Gerais, da lei que criava duas escolas de Farmácia, uma em Ouro Preto e outra em São João del Rei. A lei foi executada em parte, com a criação apenas da Escola de Farmácia de Ouro Preto, tornando-se esta a primeira faculdade de Minas Gerais e a mais antiga do gênero na América Latina. Construída na antiga sede da Assembleia Provincial, local onde foi jurada a 1ª Constituição Republicana de Minas Gerais, a Escola foi recentemente transferida para o campus Morro do Cruzeiro, em Ouro Preto.

A Escola de Minas de Ouro Preto, primeira instituição brasileira dedicada ao ensino de mineração, metalurgia e geologia, foi fundada em 12 de outubro de $1876$, pelo professor Francês Henri Gorceix, a pedido do imperador D. Pedro II. Assumiu um papel preponderante no quadro do ensino superior no Brasil na primeira metade do século XIX. Sediada no antigo Palácio dos Governadores, no centro de Ouro Preto, foi transferida em $1995$ para o campus Morro do Cruzeiro.\footnote{Entretanto, parte de suas instalações ainda é utilizada, podendo-se citar, o laboratório de eletrotécnica, único no país por ser, ao mesmo tempo, laboratório didático e museu. A antiga Escola também é utilizada pelo programa de pós-graduação em Engenharia de Materiais, além de laboratórios do curso de Museologia, pertencentes à recém $(2013)$ Escola de Turismo, Direito e Museologia (EDTM). Há ainda o Museu de Ciência e Técnica, que conta com um enorme acervo integrante dos setores de física, mineralogia e astronomia.}

Embora criada em $1969$, a Universidade Federal de Ouro Preto permaneceu até o final dos anos setenta sem uma estrutura capaz de justificar, perante o Ministério da Educação, sua manutenção como universidade. A Lei $5.540$ da Reforma Universitária de 1968 continha uma série de exigências, às quais a UFOP, somente no início dos anos $80$ começaria a se adequar. A criação dos institutos básicos é um bom exemplo dessa questão. 

Em $1978$ é criado o curso de Nutrição, sendo que a Escola de Nutrição seria efetivamente fundada apenas em $1994$, no campus Morro do Cruzeiro. Em $1979$, na cidade de Mariana (MG), seria criado o Instituto de Ciências Humanas e Sociais (ICHS). Localizado no belíssimo e histórico prédio onde funcionava o Seminário de Nossa Senhora da Boa Morte, hoje o campus abriga os cursos de História, Letras e Pedagogia.

Pouco tempo depois $(1982)$, no campus Morro do Cruzeiro, seria criado o Instituto de Ciências Exatas e Biológicas (ICEB), responsável, inicialmente, pelas disciplinas de graduação dos ciclos básicos dos cursos da Escola de Minas, Farmácia e Nutrição. Na atualidade, abrange os cursos de graduação em Ciências Biológicas, Matemática, Ciência da Computação, Estatística, Física, Química e Química Industrial. Atende também às disciplinas básicas de cursos da área da saúde, como Medicina e Educação Física.

A UFOP, naquela época, tinha como missão a formação profissional, principalmente através do ensino de graduação e o aperfeiçoamento profissional, e como valores a tradição, a endogenia, a formação profissional em detrimento da acadêmica, maior aproximação entre docentes e discentes, além da manutenção de unidades individuais, embora constituída como Universidade. Essa situação começou a mudar quando, a partir de $1990$, a UFOP implementou uma estratégia de recolocação no cenário nacional, que buscou atender a algumas pressões fundamentais: de um lado a pressão advinda de mecanismos de avaliação institucional que, em última análise, vinculam as dotações orçamentárias à posição de uma instituição dentro do quadro geral das IFES.\footnote{Nesse cenário, não crescer, ou crescer menos que a média geral do sistema implica em perda progressiva de recursos. Da mesma forma, num processo agudo de cortes de verbas nas instituições de financiamento, não crescer em qualidade também significava privar-se de recursos.} De outro, a adequação ao novo ``modelo'' dos institutos. A criação de novos cursos e o fortalecimento de unidades básicas, que inicialmente tinham como função o ``fornecimento'' de disciplinas para cursos profissionalizantes, cria mecanismos de despolarização na universidade, permitindo a introdução de novos elementos de ``tensão'' na evolução da instituição.

Finalmente são dados passos importantes no sentido de buscar atender a demandas regionais próprias. Criam-se novos espaços, como o Centro de Artes e Convenções, e projetos de formação de professores, como o Ensino a Distância, no atual Centro de Educação Aberta e a Distância (CEAD). Com ele, a Universidade implantou cursos de pós-graduação e graduação na modalidade à distância, abrangendo cerca de $90$ cidades em Minas Gerais, quatro no estado de São Paulo e oito na Bahia. Atualmente, os cursos de graduação ofertados são Administração Pública, Geografia, Pedagogia e Matemática.

A década de $1990$ seria marcada pelo aparecimento de diversos cursos, podendo-se citar o de Direito em $1993$ e o curso de turismo em $1999$ que, além de reforçar o papel da Universidade na região, promove uma visão voltada para o desenvolvimento do mercado turístico. São criados diversos cursos de engenharia, entre eles a Engenharia Ambiental, Engenharia de Produção e a Engenharia de Controle e Automação. Curso este que, aliás, foi o campeão de inscrições no vestibular do ano seguinte $(2000)$, quando a primeira turma foi aberta.

Em subsequente mas não linear processo de ampliação, a UFOP inaugura o campus avançado de João Monlevade em $2002$, oferecendo os cursos de Sistemas de Informação e Engenharia de Produção, culminando com a criação dos cursos de Engenharia Elétrica e Engenharia de Computação, em $2009$, após a adesão da Universidade ao programa de Apoio aos Planos de Restruturação das Universidades Federais (REUNI). Assim, o referido campus avançado teve seu status elevado à condição de Instituto de Ciências Exatas e Aplicadas (ICEA). 

A título de análise, vale ressaltar que o Reuni permitiu à universidade mais do que duplicar a oferta de vagas dos cursos existentes, desde as diversas Engenharias (Engenharias de Controle e Automação, Produção, Civil, Ambiental, Geológica e de Minas), criação de novos cursos, como a Arquitetura e Urbanismo na Escola de Minas a licenciatura em Educação Física, em $2008$, no Centro Desportivo da Universidade (CEDUFOP), campus Morro do Cruzeiro, unidade já existia desde os anos $1970$, tendo desenvolvido desde então parceria com vários cursos de graduação. Houve igualmente a concretização de mais uma unidade na cidade de Mariana, onde foram abrigados quatro cursos: Administração, Ciências Econômicas, Jornalismo e Serviço Social, que funcionam, desde $2008$, no Instituto de Ciências Sociais e Aplicadas (ICSA). 

No início de $2013$, foi criada a Escola de Medicina, no campus Morro do Cruzeiro, responsável por sediar o curso de Medicina. O curso, que surgiu em $2007$ e funcionava junto com o Departamento de Farmácia, agora tem prédio próprio. Outra conquista foi a implantação da graduação em Museologia, primeira de Minas Gerais. Suas atividades são realizadas também no Morro do Cruzeiro.

Atualmente, a universidade ocupa uma área de aproximadamente $151$ mil $m^2$, com mais de $150$ salas de aula e $140$ laboratórios de ensino e pesquisa. Conta, ainda, com $848$ professores efetivos e $806$ técnicos-administrativos. Oferece $51$ cursos de graduação, endo quatro na modalidade à distância:Pedagogia, Administração Pública, Licenciatura em Geografia e Licenciatura em Matemática. Em relação à pós-graduação, são ofertados $13$ programas de doutorado, $28$ de mestrado e $20$ especialização lato sensu, sendo $13$ presenciais e $7$ à distância. Quanto ao corpo discente, são $13.021$ alunos de graduação, $1.409$ deles matriculados na modalidade a distância. Na pós-graduação, são $357$ matrículas em programas de doutorado; $1.118$ em programas de mestrado, dos quais $860$ são em mestrado acadêmico e $258$ em mestrado profissional; e aproximadamente $3.500$ matrículas em programas de especialização (presencial e a distância).\footnote{De acordo com o plano de desenvolvimento institucional (PDI) aprovado para o período $(2016-2025)$ \cite{resolucao-cuni-1}.} Uma considerável diversidade, especialmente para o calouro que aporta pela primeira vez em Ouro Preto.
%
\section{Missão e valores da UFOP}
%
Além da formação tradicional em virtude dos diversos cursos de graduação, pós-graduação, pesquisa e extensão, a universidade possui uma proposta de preservação que se reafirma por meio de  projetos como a Oficina de Cantaria, que recupera importantes monumentos históricos, o Fórum das Artes, concebido com a intenção de promover o diálogo entre autor e público participante, além de valorizar a importância de Ouro Preto, Patrimônio Cultural da Humanidade, nos âmbitos turístico e cultural, por meio da valorização da identidade e da diversidade literária dos países de língua portuguesa, através da cooperação mútua entre África, Brasil e Portugal. O evento também promove um intenso intercâmbio com países latino-americanos e outros de origem latina, solidificando ainda mais a interação entre estas nações.

O Museu de Ciência e Técnica, o Museu de Pharmácia, o Observatório Astronômico e o Cine Vila Rica são importantes centros de conservação da memória e da cultura nacionais, que guardam um legado de conhecimento para a toda sociedade. Neles são desenvolvidos diversos programas de educação que buscam inserir a comunidade regional em importantes reflexões acerca dos saberes humanos. O Cine Vila Rica, aliás, continua sendo o único cinema da região.

A Universidade Federal de Ouro Preto deve se firmar como agente capaz de contribuir para a construção de uma sociedade justa, plural e pautada na sustentabilidade. É em torno desse objetivo que são definidos sua missão, visão e valores
\begin{itemize}
	\item [Missão:] Produzir e disseminar o conhecimento científico, tecnológico, social, cultural, patrimonial e ambiental, contribuindo para a formação do sujeito como profissional ético, crítico-reflexivo, criativo, empreendedor, humanista e agente de mudança na construção de uma sociedade justa, desenvolvida socioeconomicamente, soberana e democrática.
	\item [Visão:] Ser uma universidade de excelência e reconhecida pela produção e integração acadêmica, científica, tecnológica e cultural, comprometida com o desenvolvimento humano e socioeconômico do país.
	\item [Valores:] À luz dos princípios constitucionais e das finalidades estatutárias, a atuação da UFOP pauta-se nos seguintes valores:
	\subitem autonomia;
	\subitem compromisso, inclusão e responsabilidade social;
	\subitem criatividade;
	\subitem democracia, liberdade e respeito;
	\subitem democratização do ensino e pluralização do conhecimento;
	\subitem eficiência, qualidade e excelência;
	\subitem equidade;
	\subitem indissociabilidade;	
	\subitem integração e interdisciplinaridade;
	\subitem parcerias;
	\subitem preservação do patrimônio artístico, histórico e cultural; 
\end{itemize}

Além disso, a universidade tem investido na formação continuada de pessoal docente, criando diversos programas de formação, como o ``sala aberta'', que é uma idealização da pró-reitoria de graduação e do Núcleo de apoio pedagógico da UFOP, o NAP.

Seguindo a tendência das grandes universidades brasileiras, a UFOP tem levantado esforços no sentido alavancar a nova visão institucional \cite{resolucao-cuni-1} que prima pela tônica da internacionalização. Um exemplo desta iniciativa é a recente criação de disciplinas de graduação no idioma Inglês,\footnote{As primeiras turmas tiveram início no segundo semestre de $2016$.} além de programas voltados à recepção e inserção de alunos estrangeiros na universidade (ver figura \ref{fig:01}).
%
\begin{figure}[!bpt]
	\centering
	\includegraphics[scale=0.5]{figuras/Summer_Course1.jpg}
	\caption{Primeiro programa de recepção de alunos estrangeiros.}
	\label{fig:01}
\end{figure}
%
\section{Realidade regional e do aluno na UFOP}
%
Em uma estrutura \textit{multicampi}, formada pelos \textit{campi} de Ouro Preto, Mariana e João Monlevade, a universidade está inserida na mesorregião de Belo Horizonte, estendendo-se até João Monlevade, e na microrregião de Ouro Preto, que abrange as cidades de Itabirito, Ouro Preto, Mariana, Diogo de Vasconcelos e Acaiaca. Essa microrregião abarca, conforme dados do censo de $2015$, uma população de aproximadamente $180$ mil habitantes, $193$ unidades escolares estaduais e municipais, uma universidade, um instituto federal e $37$ escolas da rede privada de ensino, com um público escolar de cerca de $5$ mil profissionais da educação e $52$ mil alunos, o que demanda da UFOP uma importante inserção acadêmica e reconhecimento na região \cite{resolucao-cuni-1}.
%
\begin{comment}
A cidade histórica de Ouro Preto é famosa por sua arquitetura colonial e pelo clima peculiar, que dá um charme especial à cidade. Situada a uma altitude média de $1.179$ metros, abriga uma população de mais de $70.000$ habitantes, conforme o censo de $2010$ do Instituto Brasileiro de Geografia e Estatística (IBGE). Com mais de 300 anos de história, Ouro Preto é um dos principais símbolos de Minas Gerais, e não só entre os limites do país, mas também no exterior. A antiga Vila Rica, no passado berço de alguns dos mais importantes movimentos na luta pela independência brasileira e hoje palco de grandiosos eventos culturais, é um dos ícones máximos do Barroco nacional e mundial. A cidade, tombada como Patrimônio Histórico e Cultural da Humanidade, título conferido pela Unesco, é berço de escritores, artistas e toda sorte de personalidades. Mas sobretudo, Ouro Preto é conhecida por ser uma cidade eminentemente universitária.
\end{comment}

Os estudantes da UFOP procedem principalmente dos estados de Minas Gerais, São Paulo, Rio de Janeiro, Espírito Santo e Goiás. Uma realidade que foi sendo alterada gradativamente ao longo dos últimos $15$ anos, período em que a universidade aderiu às novas formas de gestão da educação por parte do governo federal.\footnote{Destaca-se, neste cenário, a alteração do tradicional ``vestibular'' para o Sistema de Seleção Unificada (Sisu), do ministério da educação, em que  as instituições públicas de educação superior oferecem vagas a candidatos participantes do Exame Nacional do Ensino Médio (Enem). Este exame, aliás, foi inspirado nos sistemas utilizados por outros países, como o SAT norte-americano, o Baccalauréat francês e o Gāo Kǎo chinês. O ENEM é considerado o segundo maior exame do gênero no mundo, só sendo superado pelo exame aplicado na China.} Excetuando-se os alunos provindos das escolas federais, percebeu-se um aumento significativo de egressos das escolas públicas da chamada ``região dos Inconfidentes''(Ouro Preto, Mariana e arredores), sobretudo aqueles de Escolas públicas estaduais, que eram raros há alguns anos atrás, especialmente aqueles advindos de classes sociais menos favorecidas. Isto não significa, entretanto, que a realidade da educação pública, sob a ótica do ensino médio, tenha se alterado de forma que o acesso à universidade pública fosse facilitado.

E viver como estudante em Ouro Preto significa, quase sempre, a sua integração em uma das marcas da cultura universitária da cidade: as ``repúblicas estudantis'', que são um tópico à parte.

As repúblicas fazem parte da tradição da cidade. Muitas delas instaladas em prédios históricos, pertencentes à Universidade Federal de Ouro Preto (sendo simplesmente denominadas como ``Repúblicas Federais''), absorvem parcela significativa dos estudantes, tanto em Ouro Preto quanto Mariana. São administradas pelos próprios estudantes, que definem suas regras de admissão. Ao longo de mais de um século, as repúblicas desenvolveram uma cultura própria e mantêm laços estreitos com ex-alunos e ex-residentes, inclusive no que tange à rede de contatos profissionais. Esse laço é muito forte entre as repúblicas federais e mesmo entre as chamadas ``repúblicas particulares'' (que não pertencem à universidade), especialmente na condução das famosas e tradicionais ``festas republicanas'': 12 de outubro, aniversário da Escola de Minas e o 21 de abril, dia de Tiradentes e aniversário das Repúblicas do Campus. Nestes períodos, os antigos ex-alunos costumam marcar presença como uma forma de relembrar e reviver as lembranças dos tempos de universidade. A Refop (Associação das Repúblicas Federais de Ouro Preto) é composta por $67$ repúblicas, sendo uma mista, $51$ masculinas e $15$ femininas. Os alunos veteranos e muitos ex-alunos atestam que a estrutura que existe entre os moradores de uma república são fundamentais para a sua formação como futuros profissionais, especialmente em um contexto de mundo globalizado.
%
\chapter{A gradua{\c c}{\~a}o em Engenharia de Controle e Automa{\c c}{\~a}o}
As atuais demandas da sociedade por bens e servi{\c c}os t{\^e}m sido cada vez mais atendidas utilizando-se de novas tecnologias, resultantes da aplica{\c c}{\~a}o do conhecimento cient{\'i}fico. E isto não é diferente no microclima da região dos inconfidentes, principal ``cliente'' da universidade.

O ensino de engenharia em face dessa realidade passa por grandes mudan{\c c}as, com a cria{\c c}{\~a}o de novas habilita{\c c}{\~o}es, a concep{\c c}{\~a}o e adaptação de novos curr{\'i}culos e estrat{\'e}gias pedag{\'o}gicas, com o objetivo de formar engenheiros capazes de desenvolver, aperfei{\c c}oar e utilizar as novas tecnologias de base cient{\'i}fica.
 
Com o grande desenvolvimento da eletr{\^o}nica, da inform{\'a}tica e das comunicações nas últimas d{\'e}cadas, uma das {\'a}reas mais ativas da engenharia em todo mundo passou a ser, obviamente, a {\'a}rea de Controle e Automa{\c c}{\~a}o.

\section{A importância da Engenharia de Controle e Automa{\c c}{\~a}o}
\label{sec:controleeautomacao}
A hist{\'o}ria do ramo no Brasil data de $1953$, quando o Instituto Tecnol{\'o}gico de Aeron{\'a}utica (ITA) ministrou, pela primeira vez, um curso de controle autom{\'a}tico. Desde ent{\~a}o, a {\'a}rea de Autom{\'a}tica $-$ termo criado para designar a ci{\^e}ncia e a Engenharia de Controle e Automa{\c c}{\~a}o no Brasil$-$ desenvolveu-se rapidamente nas universidades brasileiras, destacando-se os cursos de controle de sistemas din{\^a}micos da USP e Unicamp, j{\'a} no in{\'i}cio dos anos $1970$. 

Atualmente, existem cursos de engenharia associados a sistemas mecatr{\^o}nicos nos Estados Unidos, no Canad{\'a}, na Europa e na {\'A}sia. Ciente da necessidade e da relev{\^a}ncia de se formar engenheiros de Controle e Automa{\c c}{\~a}o no Brasil, o Minist{\'e}rio da Educa{\c c}{\~a}o, atrav{\'e}s da Portaria $1.694$, de $5$ de dezembro de $1994$, publicada no Di{\'a}rio Oficial da Uni{\~a}o de $12$ de dezembro de $1994$, considerando o consubstanciado no parecer da Comiss{\~a}o de Especialistas do Ensino de Engenharia da Secretaria de Educa{\c c}{\~a}o Superior, regulamentou a Engenharia de Controle e Automa{\c c}{\~a}o.

Diversas universidades brasileiras oferecem cursos associados a sistemas mecatr{\^o}nicos, como os cursos oferecidos pela Escola Polit{\'e}cnica da Universidade de S{\~a}o Paulo, o curso de Engenharia de Controle e Automa{\c c}{\~a}o da Universidade Estadual de Campinas, o curso de Engenharia Mec{\^a}nica com {\^e}nfase em Automa{\c c}{\~a}o e Sistemas da Escola de Engenharia de S{\~a}o Carlos da Universidade de S{\~a}o Paulo, o curso de Engenharia de Controle e Automa{\c c}{\~a}o da Universidade Federal de Santa Catarina, o curso de Engenharia de Controle e Automa{\c c}{\~a}o da Universidade Federal de Minas Gerais, o curso de Engenharia Mec{\^a}nica com {\^e}nfase em Mecatr{\^o}nica e o curso de Engenharia de Controle e Automa{\c c}{\~a}o da Pontif{\'i}cia Universidade Católica de Minas Gerais, o curso de Engenharia de Controle e Automa{\c c}{\~a}o da Universidade de Bras{\'i}lia e o curso de Engenharia de Controle e Automa{\c c}{\~a}o da Universidade Federal de Itajub{\'a}.

Na Universidade Federal de Ouro Preto, o curso de Engenharia de Controle e Automa{\c c}{\~a}o foi concebido no final da d{\'e}cada de $1990$ de forma a responder às necessidade de expans{\~a}o da pr{\'o}pria institui{\c c}{\~a}o (e tamb{\'e}m do mercado), face aos novos tempos. Sabia-se, at{\'e} aquele momento, que a universidade possu{\'i}a contribui{\c c}{\~o}es significativas para o desenvolvimento da engenharia no Brasil, especialmente pela tradi{\c c}{\~a}o centen{\'a}ria da Escola de Minas. Assim, em agosto do ano $2000$, iniciam-se as primeiras turmas do curso de gradua{\c c}{\~a}o em Engenharia de Controle e Automa{\c c}{\~a}o na UFOP. Vale mais uma vez ressaltar que o curso foi o campeão de inscrições na universidade, quando aberto o primeiro vestibular.

Sob a ótica industrial, não é desproposital afirmar que há, ainda, escassez de profissionais com essa formação no país. Segundo informações do guia do estudante de $2016$ \cite{guia-do-estudante}, os setores de petróleo e gás, manufatura, mineração e metalurgia são tradicionais empregadores. Três novas áreas apresentam grande potencial: indústria portuária, robótica e a domótica (pesquisa e desenvolvimento de automação de rotinas e tarefas domésticas). Empresas automobilísticas também demandam o graduado. O Sul, o Sudeste e a região da Zona Franca de Manaus são os principais centros de emprego ao longo do país. Na região de maior atuação da universidade federal de Ouro Preto, os grandes empregadores são as empresas de mineração, metalurgia e as de base tecnológica situadas na região metropolitana de Belo Horizonte. 

Pode-se inclusive, ressaltar a crescente importância da Engenharia de Controle e Automação para as áreas de:
\begin{itemize}
\item conservação do patrimônio histórico, especialmente na área de conforto térmico: controle de temperatura, umidade e iluminação;
\item iluminação pública;
\item automação comercial e predial;
\item pesquisa e desenvolvimento de tecnologias assistivas para portadores de necessidades especiais: visão, motora e intelectual;
\item automação agrária.
\end{itemize}

Desde a criação da graduação em Engenharia de Controle e Automação, que comemora seus quase $16$ anos de história, poucas alterações haviam ocorrido em seu arcabouço. E isto vai de encontro a um mundo que se alterou rapidamente, em termos científicos, econômicos e tecnológicos no mesmo período.

Após a divulgação, por parte da pró-reitoria de graduação da UFOP de um programa de restruturação chamado Plano de Ações Pedagógicas (PAP) em $2014$, surgiu a necessidade de se rediscutir toda a base da graduação em Engenharia de Controle e Automação. E isto passa pela rescrita de todo o projeto de curso e não apenas sua atualização.

\section[O PPC]{O projeto pedag{\'o}gico de curso} \label{sec:ocurso}
%
A principal proposta do Projeto Pedagógico de Curso ou, simplesmente, PPC\footnote{Podendo ser chamado também de Projeto Político e Pedagógico, PPP.} {\'e} a de envolver discentes, docentes e t{\'e}cnicos administrativos, no contexto do curso de engenharia de controle e automa{\c c}{\~a}o, para conscientiza{\c c}{\~a}o  e participa{\c c}{\~a}o ativa frente aos desafios a serem superados, que culminem na restrutura{\c c}{\~a}o e aperfei{\c c}oamento da graduação como um todo. Entende-se que este n{\~a}o deve ser mais um documento, mas um horizonte norteador que se prop{\~o}e a direcionar o trabalhos e a vis{\~a}o de toda a comunidade acad{\^e}mica. 

No primeiro ciclo de debates, promovidos pelo N{\'u}cleo Docente Estruturante (NDE) em parceria com o colegiado de curso (CECAU), foram levantadas diversas propostas para a melhoria cont{\'i}nua, seguindo a filosofia proposta pela PROGRAD, em conson{\^a}ncia com o projeto de desenvolvimento institucional da Universidade Federal de Ouro Preto (PDI-UFOP). A ideia geral foi a de elaborar a{\c c}{\~o}es de car{\'a}cter mais espec{\'i}fico, que repercurtir{\~a}o em curto, m{\'e}dio e longo prazos.

O consenso geral foi a necessidade de consolidar a gradua{\c c}{\~a}o em engenharia de controle e automa{\c c}{\~a}o, por interm{\'e}dio da atualiza{\c c}{\~a}o em seu projeto pedag{\'o}gico (PPC), especialmente ap{\'o}s a chegada de novos docentes e o aumento (de $100\%$) na oferta de vagas para alunos ingressantes a cada ano, ap{\'o}s o programa de restrutura{\c c}{\~a}o das universidades fererais, o Reuni. 

Tomou-se ent{\~a}o como principais desafios para restrutura{\c c}{\~a}o do PPC a atualiza{\c c}{\~a}o da matriz curricular, com a revis{\~a}o de programas de disciplinas, identifica{\c c}{\~a}o e diminui{\c c}{\~a}o do d{\'e}ficit de bibliografias, elabora{\c c}{\~a}o de regras internas para a capacita{\c c}{\~a}o de docentes, cria{\c c}{\~a}o de material instituicional de divulga{\c c}{\~a}o, inser{\c c}{\~a}o de calouros e envolvimento de veteranos em estudos de novas pr{\'a}ticas pedag{\'o}gicas, como a ``aprendizagem ativa'', o aumento no n{\'u}mero de bolsas de monitoria e at{\'e} mesmo quest{\~o}es relativas {\`a} infra-estrutura de laborat{\'o}rios, que seguramente repercutir{\~a}o na qualidade do ensino e na diminui{\c c}{\~a}o das taxas de evas{\~a}o do curso.
%\chapter{Concepção do Curso}

\section{Dados de identificação do curso}

\section{Objetivos e perfil do egresso} \label{sec:perfildoegresso}

Poder-se-ia dizer que a forma{\c c}{\~a}o do engenheiro de controle e automa{\c c}{\~a}o encerra um car{\'a}ter abrangente de atua{\c c}{\~a}o na natureza e, por este motivo,  tem como pedra fundamental a integra{\c c}{\~a}o entre diversas {\'a}reas do conhecimento humano, desde a matem{\'a}tica, f{\'i}sica, qu{\'i}mica e ci{\^e}ncias da computa{\c c}{\~a}o. No entanto, sua atua{\c c}{\~a}o n{\~a}o se restringe a tais campos do saber, podendo dar-se em inumer{\'a}veis outros campos, o que torna dif{\'i}cil a tarefa de enquadr{\'a}-lo nas tradicionais {\'a}reas da engenharia, como el{\'e}trica, eletr{\^o}nica, mec{\^a}nica ou computa{\c c}{\~a}o.

Os departamentos de Matem{\'a}tica, F{\'i}sica, Computa{\c c}{\~a}o, Controle e Automa{\c c}{\~a}o e T{\'e}cnicas Fundamentais(necessita de revis{\~a}o para incluir a Mec{\^a}nica), Metalurgia e de Materiais, Minas e Produ{\c c}{\~a}o disp{\~o}em de recursos humanos, instala{\c c}{\~o}es e desenvolvem atividades de ensino e de pesquisa de alto n{\'i}vel nas {\'a}reas de Ci{\^e}ncias Exatas, Ci{\^e}ncias da Engenharia e Processos Industriais em Minera{\c c}{\~a}o e Metalurgia. 

Al{\'e}m disso, o Departamento de Computa{\c c}{\~a}o desenvolve atividades de ensino e de pesquisa em todas as {\'a}reas de forma{\c c}{\~a}o profissional da Engenharia de Controle e Automa{\c c}{\~a}o ligadas a software e o Departamento de Engenharia de Controle e Automa{\c c}{\~a}o e T{\'e}cnicas Fundamentais(necessita de revis{\~a}o para incluir a Mec{\^a}nica) tem um núcleo consolidado que desenvolve ensino e pesquisa tecnológica em todas as {\'a}reas de forma{\c c}{\~a}o profissional ligadas a hardware do Curso de Engenharia de Controle e Automa{\c c}{\~a}o.

Portanto, as necessidades de investimento em recursos materiais e na contrata{\c c}{\~a}o de novos profissionais para a implanta{\c c}{\~a}o do Curso de Engenharia de Controle e Automa{\c c}{\~a}o na UFOP s{\~a}o complementares e relativamente pequenas, ficando demonstrada a viabilidade de um Curso de Engenharia de Controle e Automa{\c c}{\~a}o na UFOP.
 

\section{Situa{\c c}{\~a}o atual e desafios}

Pode-se afirmar que, desde a cria{\c c}{\~a}o do curso de eng. de controle e automa{\c c}{\~a}o em $1999$, muitos desafios foram superados. Identificou-se como sendo um dos principais focos dados para o curso at{\'e} ent{\~a}o, especialmente no que concerne ao perfil dos egressos, o da forma{\c c}{\~a}o espec{\'i}fica para a atua{\c c}{\~a}o em empresas, destacando-se aquelas do setor m{\'i}nero-metal{\'u}rgico. Tal justificativa dava-se pela voca{\c c}{\~a}o da pr{\'o}pria regi{\~a}o e necessidades do mercado. No entanto, passada mais de uma d{\'e}cada de sua exist{\^e}ncia, detectamos a necessidade de expans{\~a}o de conceitos, de forma a contribuir na forma{\c c}{\~a}o de engenheiros com perfil mais abrangente e empreendedor, dentro de uma perspectiva hol{\'i}stica, {\'e}tica e humanista, e n{\~a}o apenas para suprir demandas mercadol{\'o}gicas.



%A partir de tais discuss{\~o}es, o primeiro plano de a{\c c}{\~o}es para o curso de Engenharia de Controle e Automa{\c c}{\~a}o foi elaborado e sumarizado na planilha anexa